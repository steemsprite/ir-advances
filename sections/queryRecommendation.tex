\section{Query Recommendation}

% următoarele 2 slide-uri merg aici pe post de introducere? sau separat?

\begin{frame}{Human Perception of Query Quality}

\begin{quote}
How do users rate the quality of a query?
\end{quote}

\vspace{10pt}

Judging by search problem / query features:
	\begin{itemize}
		\item high specificity
		\item longer queries (4-5 words)
		\item comparing to what query they would personally use
		\item queries for fact-finding problems rated lower than\\ queries for
		exploratory problems % formulare prea verbose
		\begin{itemize}
			\item fact-finding: height of the Eiffel Tower
			\item exploratory: wildlife in the Masoala National Park
		\end{itemize}
	\end{itemize}
After viewing a results page:
	\begin{itemize}
		\item number of relevant results out of those viewed % pe grafic
		\item result rank is not as good an indicator
	\end{itemize}

\end{frame}

\begin{frame}{Query Quality vs Result Relevance}
\begin{figure}
	\includegraphics[width=0.8\textwidth]{img/qquality.png}
\end{figure}
Difference in query rating \textbf{before} and \textbf{after} viewing the result page
\begin{itemize}
	\item 0 relevant results clicked: lower rating
	\item 1-2 relevant results clicked: undetermined
	\item 3 or more relevant results clicked: higher rating
\end{itemize}
\end{frame}

\begin{frame}{Query Recommendation}

Suitable refinements to an initial query

\end{frame}


\begin{frame}{Query Recommendation on Intranets}

Document collection on Intranet:
\begin{itemize}
	\item relatively small
	\item changes less frequently (than Internet)
	\item more limited context of search (than Internet) \newline
\end{itemize}

Methods for Query Recommendation on Intranets:
\begin{itemize}
	\item concept hierarchy
	\item search logs
\end{itemize} 

\end{frame}


\begin{frame}{The Concept Hierarchy Model (CHM)}

What is a concept?
\begin{itemize}
	\item keyword
	\item phrase
	\item n-grams \newline
\end{itemize}

Concept Hierarchy:
\begin{itemize}
	\item created from a document collection
	\item used for ranking terms according to the strength of their links to the query within the hierarchy
\end{itemize}

\end{frame}


\begin{frame}{Subsumption Hierarchy for REsult Clustering (SHReC)}

Use term co-occurence to build a subsumption hierarchy tree \newline

For each concept (node in graph):
\begin{itemize}
	\item all direct ascendants are generalizations for the current concept
	\item all direct descendants are specializations for the current concept \newline
\end{itemize}

Static hierarchy with a high computational cost to frequently rebuild it 

\end{frame}


\begin{frame}{Query Recommendation using Search Logs}

Types:
\begin{itemize}
	\item local (based only on each user's previous searches)
	\item global (based on all previous searches) \newline
\end{itemize}

Query Flow Graph
\begin{itemize}
	\item state-of-the-art log-based query recommender
	\item directed graph:
		\begin{itemize}
			\item nodes contain queries
			\item edges represent query refinements \newline
		\end{itemize}
\end{itemize}

Dynamic structure and easy to be updated

\end{frame}


\begin{frame}{Adaptation of CHM with Search Logs}

Concept Hierarchy discovers relationships between terms based on statistical analysis, giving a conceptual view \newline

Search Logs provide a conceptual view based on collective user intelligence \newline

Two complementary conceptual views on the same search domain
$ \Rightarrow $
Adapt the CHM with user interactions derived from Search Logs

\end{frame}


\begin{frame}{SHReC Results}

Adaptive SHReC is more efficient than QFG, which is more efficient that static SHReC \newline

Users will benefit more from the models when all log refinements are used for adaptation rather than only those with clicks

\end{frame}


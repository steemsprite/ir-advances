\subsection{Implicit Feedback Analysis}

\begin{frame}{Implicit Feedback}
  Explicit feedback
  \begin{itemize}
  \item Ask users to rank results
  \end{itemize}

  Implicit feedback
  \begin{itemize}
  \item \textbf{On what links did the user click?}
  \item Eye Tracking
  \end{itemize}


  Is implicit feedback useful?
  \begin{itemize}
  \item Pros
    \begin{itemize}
    \item Easier to obtain in large quantities
    \item No burden on the user
    \end{itemize}
  \item Cons
    \begin{itemize}
    \item More difficult to interpret
    \item Potentially noisy
    \end{itemize}
  \end{itemize}
\end{frame}

\begin{frame}{Implicit Feedback Experiment}
  \begin{itemize}
  \item Can we infer anything from click patterns? \newline
  \item Suppose we invent a new algorithm to do so. \newline
  \item How to check if it's correct? \newline
    \begin{enumerate}
    \item Ask users to perform some search tasks using Google
    \item Gather both explicit and implicit feedback
    \item Apply the algorithm on the implicit feedback
    \item Validate the results using explicit feedback
    \end{enumerate}
  \end{itemize}
\end{frame}

%\begin{frame}{User Study}
%  \begin{itemize}\addtolength{\itemsep}{1.0\baselineskip}
%    \item Some general facts (obtained from the study)
%      \begin{itemize}\addtolength{\itemsep}{1.0\baselineskip}
%      \item Users click more often on the first link
%      \item The first two links receive equal attention (number of fixations)
%      \item Users tend to scan the results from top to bottom
%      \end{itemize}
%  \end{itemize}
%\end{frame}

\begin{frame}{Analyzing Clicks}
  Are clicks absolute relevance judgments?
  \begin{itemize}
  \item Short answer: No
  \item Trust Bias
    \begin{itemize}
    \item Users trust the search engine
    \item First link receives the most clicks
      (Even if the first two results are swapped without user knowledge)
    \end{itemize}
  \item Quality-of-Context Bias
    \begin{itemize}
    \item If the relevance of the results decreases, users click on links
      that are on average less relevant
    \end{itemize}
  \end{itemize}
  Clicks are only relative relevance judgments
  \begin{itemize}
  \item If a link was clicked it means it was relevant for the user,
    not relevant on an absolute scale
  \end{itemize}
\end{frame}

\begin{frame}{Analyzing Clicks (2)}
  Suppose we have the following links (* means the link was clicked): \newline
  \begin{itemize}
  \item $L_1^{*}$, $L_2$, $L_3^{*}$, $L_4$, $L_5^{*}$, $L_6$, $L_7$
  \end{itemize}
  We can extract a \textbf{partial} ordering relation (rel = relevance)
  \begin{itemize}
    \item $rel(L_3) > rel(L_2)$, $rel(L_5) > rel(L_2)$, $rel(L_5) > rel(L_4)$
    \item We call this ``click $>$ skip above''
    \item Other strategies:
      \begin{itemize}
      \item last click $>$ skip above
      \item click $>$ earlier click
      \item click $>$ skip previous
      \item click $>$ no-click next
      \end{itemize}
    \item Similar strategies for query chains
  \end{itemize}
\end{frame}

\begin{frame}{Analyzing Clicks (3)}
  \begin{itemize}
  \item Check if these strategies are consistent with explicit feedback
  \item Result: \textbf{yes}, most of them are (with $> 80\%$ accuracy)
    \begin{itemize}
    \item click $>$ skip above
    \item last click $>$ skip above
    \item click $>$ skip previous
    \item click $>$ no-click next
    \end{itemize}
  \end{itemize}
\end{frame}

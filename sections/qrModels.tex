\subsection{Generalized Models of Query Reformulation}

\begin{frame}
	\frametitle{Query Reformulation}
	\begin{block}
	 {The process of iteratively modifying a query to improve
the quality of a search engine results}
	\end{block}
	\begin{itemize}	
	 \item Explicitly: suggesting related queries or query completions
	 \item Implicitly: query expansion
	\end{itemize}
	\vskip5pt
	Reformulations are close to the previous query both:
	\begin{itemize}
	 \item \textit{syntactically}, as sequence of characters and terms
	 \item \textit{semantically}, involving taxonimic relations (generalization, specialization).
	\end{itemize}
	\textbf{Goal}: Models for query reformulation which combine the syntactic and
the semantic aspects (generalized models)

\end{frame}

\begin{frame}
	\frametitle{Semantic similarity measures}
	\begin{columns}
	 \column{.6\textwidth}
	  \structure{PMI (pointwise mutual information)}
	 \column{.4\textwidth}
	  \begin{equation}
PMI(x,y)=\log{\frac{p(x,y)}{p(x)p(y)}}
	  \nonumber
	  \end{equation}
	\end{columns}
	\begin{columns}
	 \column{.6\textwidth}
\textbf{Joint} - shared information between two strings
	 \column{.4\textwidth}
	  \begin{equation}
PMI(J)(x,y)=\frac{PMI(x,y)}{-\log{p(x,y)}}
	  \nonumber
	  \end{equation}
	\end{columns}
	\begin{columns}
	 \column{.6\textwidth}
\textbf{Specialization} - the second string is a specialization of the first one
	 \column{.4\textwidth}
	  \begin{equation}
PMI(S)(x,y)=\frac{PMI(x,y)}{-\log{p(x)}}
	  \nonumber
	  \end{equation}
	\end{columns}
	\begin{columns}
	 \column{.6\textwidth}
\textbf{Generalization} - the second string is a generalization of the first one
	 \column{.4\textwidth}
	  \begin{equation}
PMI(G)(x,y)=\frac{PMI(x,y)}{-\log{p(y)}}
	  \nonumber
	  \end{equation}
	\end{columns}
	\vskip7pt
	$\bullet$ x="apple", y="mac os": PMI(G)=0.2917, PMI(S)=0.3686 $\Rightarrow$ specialization \newline
	$\bullet$ x="ferrari models", y="ferrari": PMI(G)=1, PMI(S)=0.558 $\Rightarrow$ "perfect"generalization
\end{frame}

\begin{frame}
	\frametitle{Syntactic similarity measures}
	\textbf{The Levenshtein distance(edit distance)} between x and y is the cost of the least expensive sequence of edit operations which transform x into y.
	\vskip7pt

	$\omega=(a_{1},b_{1}),...,(a_{n},b_{n})$ - sequence of edit operations (insert, delete, substitute) \newline
	c($\omega$)=$\sum_{i=0}^n c$($\omega_{i}$) - cost of the sequence

	\vskip7pt
	Examples of cost functions:
	\begin{itemize}
	 \item
          - Edit1: unit costs for each of the three edit operations
	  $\forall a,b, c_{E_{1}}(a,b)=1, if\ {a}\neq{b}, 0\ otherwise$
	 \item
	 - Edit2: unit costs for insertion and deletion, edit distance for substitution (if two items are similar at character level, the cost of substitution is lower) 
	\end{itemize}
\end{frame}

\begin{frame}
	\frametitle{Generalized edit distance}
	\begin{block}
	{Extension of Levenshtein to take into consideration semantic similarities}
	\end{block}
	\textbf{GenEdit}:
$\forall a,b, c_{GE}(a,b)=s(a,b), if {{a}\wedge{b}}\neq{\xi},\ 1 \ otherwise$ where
s(a,b) = 2 - 2f(a,b) + $\epsilon$, S=\textit{cost matrix},\newline
f = normalized semantic similarity function (e.g. PMI(S)),\newline
$\epsilon$ = correction \newline
	\textbf{Properties}:
	\begin{itemize}
	\item it is cheaper to substitute two semantically associated terms than deleting one and inserting the other one
	\item for an unrelated term pair, a combination of insertion and deletion is cheaper than a substitution
 	\end{itemize}
\end{frame}

\section{Context-Aware Ranking}


\begin{frame}{Context-Aware Ranking}

Main problems:
\begin{itemize}
	\item how can we take advantage of different types of contexts in ranking?
	\item how can we integrate context information into a ranking model? \newline
\end{itemize}

Context contains:
\begin{itemize}
	\item queries asked before the current query in the same session
	\item answers to those queries (clicked or skipped by the user) \newline
\end{itemize}

Example: ({\color{red} ar putea sa lipseasca, sa fie amintit de prezentator}) \newline
	if a user makes a query “jaguar” after searching “mercedes” then it is very likely to seek information about a Jaguar car rather than a jaguar as an animal

\end{frame}


\begin{frame}{Context-Aware Ranking Principles}

Relations between queries in the same session:
\begin{itemize}
	\item unrelated
	\item reformulation
	\item specialization
	\item generalization
	\item general association
\end{itemize}

\end{frame}


\begin{frame}{Reformulation}

Example: \newline
'homes for rent in atlanta' $ \rightarrow $ 'houses for rent in atlanta' \newline

Why? \newline
- search results of the previous query do not or only partially fulfill the information need \newline

Principle:
\begin{quotation}
for consecutive queries $ q_{t-1}q_{t} $ in a session such that $ q_{t} $ reformulates $ q_{t-1} $, if a search result \emph{r} for $ q_{t-1} $ is clicked on or skipped, \emph{r} as a result for $ q_{t} $ is unlikely to be clicked on and thus should be demoted
\end{quotation}

\end{frame}


\begin{frame}{Specialization}

Example: \newline
'time life music' $ \rightarrow $ 'time life Christian CD' \newline

Why? \newline
- see more specific results \newline

Principle:
\begin{quotation}
for consecutive queries $ q_{t-1}q_{t} $ in a session such that $ q_{t} $ specializes $ q_{t-1} $, the user likely prefers the search results specifically focusing on $ q_{t} $
\end{quotation}

Implementation: \newline
- promote the results matching $ q_{t} \setminus q_{t-1} $ in the set of answers to $ q_{t} $

\end{frame}


\begin{frame}{Generalization}

Example: \newline
'free online Tetris' $ \rightarrow $ 'Tetris game' \newline

Why? \newline
- get information not covered by the previous query \newline

Principle:
\begin{quotation}
for consecutive queries $ q_{t-1}q_{t} $ in a session such that $ q_{t} $ generalizes $ q_{t-1} $, the user would likely not prefer the search results specifically focusing on $ q_{t-1} $
\end{quotation}

Implementation: \newline
- demote the results matching $ q_{t-1} \setminus q_{t} $ in the set of answers to $ q_{t} $

\end{frame}


\begin{frame}{General Association}
\begin{small}

Example: \newline
'Xbox 360' $ \rightarrow $ 'FIFA 2010' \newline

Why? \newline
- when an ambiguous query is generally associated with its context, the context may help to narrow down the user's search intent \newline

Principle:
\begin{quotation}
for consecutive queries $ q_{t-1}q_{t} $ in a session such that $ q_{t} $ and $ q_{t-1} $ are generally associated, the user likely prefers the search results related to both $ q_{t-1} $ and $ q_{t} $
\end{quotation}

Implementation:
\begin{itemize}
	\item choose a topic taxonomy (such as Open Directory Project: http://www.dmoz.org)
	\item let $ C_{t-1} $ and $ C_{t} $ be the sets of topics of $ q_{t-1} $ and $ q_{t} $, and $ C_{\cap} $ the set of common topics between $ C_{t-1} $ and $ C_{t} $; promote a search result \emph{r} if the set of topics of \emph{r} shares at least one topic with $ C_{\cap} $
\end{itemize}

\end{small}
\end{frame}


\begin{frame}{Learning to rank}

Supervized machine learning problem in which the goal is to automatically construct a ranking model from training data \newline

RankSVM – learn a SVM model for classification on the preference between a pair of documents \newline

Derive features from the ranking principles and incorporate them into the RankSVM model

\end{frame}


\subsection{Verbose Queries}

\begin{frame}
\frametitle{Verbose Queries}
\begin{block}
	{Long queries in which people use many words to say what could have been expressed in a few keywords.}
\end{block}
%\begin{itemize}
%\item
$\bullet$
	Small but significat part of the query stream in web search.\\
	e.g. The average length of the MSN queries(15 million in a month) is 2.4 words. 10\% are 5 words or longer.\\
%\item
\vskip10pt
$\bullet$
	Current search engines do not perform well with long queries.\\
%\item
\vskip10pt
$\bullet$
	The most effective approach is to simply reduce the length:
	\begin{enumerate}
	\item
		stopword removal
	\item
		phrase detection
	\item
		key concept identification
	\item
		stop structure removal
	\end{enumerate}
%\end{itemize}
\end{frame}

\begin{frame}
\frametitle{Stopword removal}
\begin{block}
	{Removing all words in the query which occur on the stopword list}
\end{block}
\begin{itemize}
\item
the standard INQUERY stopword list \\
e.g. \textit{Can I work while study in Europe} $\rightarrow$ \textit{work study Europe}	
\item
automatically constructed collection dependent stopword list (frequency weighting methods)\\
Problem: Common query phrases such as \textit{"high blood pressure"} occur as frequently as some stop words.
\end{itemize}
\vskip10pt
This technique can change the meaning of the query:\\
e.g. \textit{i would like to know the origin of the phrase to be or not to be} $\rightarrow$ \textit{know origin phrase}
\end{frame}

\begin{frame}
\frametitle{Noun phrase detection}
Verbose queries are constructed as phrases or sentences.\\
\vskip10pt
Noun phrases can be automatically identified (e.g. MontyLingua toolkit).\\
e.g. noun phrases for \textit{give me a name for my next horror film}: \textit{"me"}, \textit{"a name"}, \textit{"my next horror film"}\\
\vskip10pt
Query reformulation techniques:\\
\begin{enumerate}
\item
wrap each of the noun phrases in quotation marks, without removing words:\\
\textit{give "me" "a name" for "my next horror film"}
\item
remove all words which are not part of a detected word phrase:\\
\textit{me a name my next horror film}
\end{enumerate}
\end{frame}

\begin{frame}
\frametitle{Key concept detection}
\begin{block}
{Key concept = short set of sequential words that express an important idea contained within the query}
\end{block}
Automaticallly identified with classifiers including features like: inverse document frequency, weighted information gain etc.\\
e.g. key concepts for \textit{"i read that ions cant have net dipole moments why not"}: \textit{i}, \textit{ions}, \textit{net dipole moments}\\
\vskip10pt 
Similar reformulation techniques:\\
\begin{enumerate}
\item
wrap each key concept in quotation marks, without removing words:\\
\textit{"i" read that "ions" cant have "net dipole moments" why not}
\item
remove all words which are not part of a detected key concept:\\
\textit{i ions net dipole moments}
\end{enumerate}

\end{frame}

\begin{frame}
\frametitle{Stop structure}
\begin{block}
{Stop phrase that begins at the first word in the query.}
stop phrase =  phrase that does not provide any information about the information need
\end{block}
\vskip10pt
The stop structure can be automatically detected with a sequential classifier.
\vskip10pt
Query reformulation: stop structure removed from query\\
\vskip10pt
e.g. \textit{"if i am having a lipid test can i drink black coffee"}\\
stop phrases: \textit{"if i am having a"}, \textit{"can i"}\\
query reformulation: \textit{"lipid test can i drink black coffee"}

\end{frame}

